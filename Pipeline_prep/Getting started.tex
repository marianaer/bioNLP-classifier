\documentclass[12pt]{article}
\usepackage{amsmath}
\usepackage{graphicx}
\usepackage{hyperref}
\usepackage[latin1]{inputenc}


\title{Preprocesamiento bioNLP}
\author{DEWY Team}
\date{09/10/18}


\begin{document}
\maketitle



\section{Eliminar texto no informativo}
\begin{itemize}
\item 1.1 Eliminar PMId. En shell:
\begin{verbatim}
cut -f3,4 sentences_RI_RIGC.txt
cut -f3,4 sentences_Other.txt
\end{verbatim}

\item 1.2. Remover referencias a tablas, figuras, artículos, revistas, etc.
Se utilizó un script de python para ambos archivos:
\begin{verbatim}
import re 
salida=open('/home/mescobar/Escritorio/Tercer_Semestre/Bioinfo
/senOth/Other_2.0.txt','w')

with open ('/home/mescobar/Escritorio/Tercer_Semestre/Bioinfo
/senOth/Other_0.5.txt','r') as archivo:

#Busca cualquier paréntesis con un número contenido, que 
en la mayoría de los casos son referencias, "et al" 
"Molecular microbiology" y lo sustituye por nada
    for line in archivo:
        line=re.sub(r"\(.*\d+.+?\)", "", line) 
        line=re.sub(r"et al", "", line, flags=re.IGNORECASE) 
        line=re.sub(r"molecular microbiology", "", line, 
        flags=re.IGNORECASE)
        
        salida.write(line)
salida.close()
\end{verbatim}


\end{itemize}


\section{Unir en un mismo archivo}
Unir ambos archivos libres de texto no informativo en el mismo archivo. En shell:
\begin{verbatim}
cat RI_2.0.txt Other2.0.txt > RI_Other_Cat.txt
\end{verbatim}

\section {Lematización y POS tagging}
Lematización y POS tag del archivo con ambos tipos de oraciones. En CoreNLP:


\begin{verbatim}
./corenlp.sh -annotators tokenize,ssplit,pos,lemma,ner,parse,depparse 

outputFormat conll -file  /home/mescobar/RI_Other_Cat.txt

-outputDirectory /home/mescobar/RI_Other_Cat.conll

\end{verbatim}

Esto nos regreza la lematización y POS tagging para ambos tipos de oraciones, lo cual ya es informativo para poder vectorizarse.

\end{document}

